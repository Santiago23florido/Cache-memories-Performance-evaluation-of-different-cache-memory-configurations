\section{Efficacité surfacique}

\subsection{Paramètres de cache par défaut}
Observant le fichier \texttt{cache.cfg}, on constate que la capacité totale des données pouvant être stockées dans le cache, sans compter les métadonnées telles que les \emph{tags}, identifiée comme la taille du cache, est de 131072~bytes, soit 128~KiB. De son côté, l’unité minimale chargée depuis la mémoire ou depuis le niveau \texttt{L2} vers le cache correspond à la taille de bloc, qui est de 64~bytes. La configuration standard comporte 2~voies pour le placement des blocs à l’intérieur des ensembles (\emph{sets}). Enfin, la technologie utilisée par défaut est de 0{,}090~$\mu$m, soit 90~nm.
\subsection{Taille des deux coeurs (hors caches L1).}
Les surfaces proviennent des sorties CACTI (\texttt{cacti/result\_L1\_*}) via la ligne
\texttt{Cache height x width (mm)}. 
En supposant les tailles du Tableau~12 : L1I = L1D = 32\,kB (A7) et 32\,kB (A15), on obtient :

\begin{table}[H]
\centering
\footnotesize
\resizebox{\linewidth}{!}{%
\begin{tabular}{lccc}
\hline
\textbf{Cœur} & $S_{L1I}$ (mm$^2$) & $S_{L1D}$ (mm$^2$) & $S_{L1}=S_{L1I}+S_{L1D}$ (mm$^2$) \\
\hline
A7 (32\,kB)  & 0{,}06644 & 0{,}06644 & 0{,}13287 \\
A15 (32\,kB) & 0{,}03319 & 0{,}03319 & 0{,}06639 \\
\hline
\end{tabular}
}
\caption{Surfaces des caches L1 (I et D) pour une taille de 32\,kB.}
\end{table}

Avec $S_{\text{core+L1}}(A7)=0{,}45$ mm$^2$ et $S_{\text{core+L1}}(A15)=2$ mm$^2$ :
\[
\begin{aligned}
\%L1(A7) &= \frac{0{,}13287}{0{,}45}\times 100 = 29{,}52\%, \\
\%L1(A15) &= \frac{0{,}06639}{2}\times 100 = 3{,}32\%.
\end{aligned}
\]
\[
\begin{aligned}
S_{\text{core hors L1}}(A7) &= 0{,}45-0{,}13287=0{,}31713~\text{mm}^2, \\
S_{\text{core hors L1}}(A15) &= 2-0{,}06639=1{,}93361~\text{mm}^2.
\end{aligned}
\]

 Le A7 consacre une fraction beaucoup plus importante de sa surface au L1
($\approx 29{,}52\%$) que le A15 ($\approx 3{,}32\%$), ce qui reflète un cœur plus compact à taille de L1 comparable, le poids surfacique est plus élevé sur A7 et aussi Sur le Cortex-A15, la logique complexe (ex\'ecution \emph{Out-of-Order}, renommage, etc.) domine massivement. Les caches L1 sont proportionnellement petits.
.

\subsection{Surfaces totales du Cortex A7 et du Cortex A15 et l’efficacité surfacique de chaque processeur.}

En disposant d\'ej\`a des conditions relatives \`a la surface du c\oe{}ur (hors cache), il est possible, \`a l’aide de CACTI, de d\'eterminer la surface du cache L2 en configurant les param\`etres de simulation afin de d\'efinir un L2 de 512~kB. De plus, on simule \'egalement avec CACTI afin d’estimer la surface du cache L1 pour l’ensemble de ses dimensions, et ce, pour les deux microprocesseurs.

En appliquant l’\'equation \eqref{eq:surface_systeme}, on obtient la surface totale du syst\`eme :
\begin{equation}
S_{\text{syst\`eme}} = S_{\text{c\oe{}ur\_nu}} + \left(2 \times S_{L1,\ \text{variable}}\right) + S_{L2,\ \text{fixe}}
\label{eq:surface_systeme}
\end{equation}

\input{sections/SurfacesL1L2.tex}

En analysant l’efficacit\'e surfacique des deux Cortex, on constate que, pour l’ensemble des algorithmes, l’efficacit\'e surfacique est sup\'erieure sur le Cortex-A15. De plus, aussi bien pour le Cortex-A7 que pour le Cortex-A15, l’efficacit\'e surfacique conserve un comportement similaire \`a celui analys\'e pour l’IPC sur les deux processeurs : sur Cortex-A7, on observe une croissance jusqu’\`a 8~kB de cache, ce qui correspond \`a un point d’inflexion, puis, \`a partir de cette valeur, la saturation de la croissance de l’IPC identifi\'ee pr\'ec\'edemment se traduit par des diminutions de l’efficacit\'e surfacique (comme dans le cas de Blowfish) ou, \`a d\'efaut, par une stagnation de l’efficacit\'e surfacique (comme dans le cas de Dijkstra), dans la mesure o\`u la saturation n’est pas encore atteinte de mani\`ere parfaitement ponctuelle.

De son c\^ot\'e, pour le Cortex-A15, la dynamique de croissance de l’IPC se traduit \'egalement par des augmentations de l’efficacit\'e surfacique pour l’ensemble des applications, et ce jusqu’\`a 32~kB de cache. Ces \'el\'ements renforcent les choix de configuration propos\'es, lesquels avaient \'et\'e initialement \'etablis \`a partir de l’analyse de l’IPC dans la section pr\'ec\'edente.
